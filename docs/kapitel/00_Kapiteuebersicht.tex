% Compile main-file when hitting typeset in this file
%!TEX root = ../MasterThesis.tex
%------------------------------------------------------------------------------------

\
\chapter{Einführung}
	\section{Problemformulierung}
	\section{Zielsetzung}


\chapter{Konzeptionierung}
	
	\section{Grundlegende Annahmen und deren mathematische Zusammenhänge}
		\subsection{Signalmodell}
			\todo[inline]{Beschreibung des Signalmodels wie im Buch von Benisty
			sowie Einführung der Variablen,Single-Source reverberant model, MIMO-System}
			\missingfigure{Ein erklärendes Bild zum Signalmodel - Sprecher, 
			Sensoren, Echo, Rauschen \dots}

		\subsection{Sensormodell}
		\todo[inline]{Geometrie, Wellenausbreitung (ebene Welle, Fernfeld),
		Koordinatensystem,Mikrofonvektoren, Entwicklung 3D-Arrays wie in
		\texttt{chap2-MIC.pdf}}
		\missingfigure{Ein erklärendes Bild zum Mikrofonmodel, Koordinatensystem,
		Projektion der DOA durch ebene Welle}

	\section{Schätzungsverfahren}
		\subsection{Kreuzkorrelation}
			\subsubsection{Schnelle Kreuzkorrelation}
				\todo[inline]{Erklärung der schnellen Korrelation mittels FFT,
				Zeropadding, FFT-Shift sowie Hinweis auf Buch von Martin Werner
				\cite{Book_SigSys_Werner}}
				\missingfigure{Man könnte dazu noch eine passende Grafik 
				(Zeropadding, FFT-Shift) erstellen}
		\subsection{Mehrkanal-Kreuzkorrelationskoeffizient (MCCC)}

	\section{Mikrofonarray}
		\subsection{Untersuchung des kugelförmigen Mikrofonarrays}
			\subsubsection{Schallausbreitung an schallharter Kugel}
			
		\subsection{Mikrofonarray Neudesign/Konstruktion}
			\subsubsection{Räumlich-Äquidistantes Mikrofonarray}
			\subsubsection{Räumlichs äquiwinkel Mikrofonarray}
			\subsubsection{Räumliches Aliasing}
			\subsubsection{Winkelauflösung}
			

		
	\section{Simulation}
		\subsection{Erstellung synthetischer Signale}
		\subsection{Optimierungsverfahren}
		\subsubsection{Suchgeschwindigkeit}
		\subsubsection{Histogramm}
		\subsubsection{Ausschlussverfahren}
	
	
\chapter{Realisierung}

	\section{Implementierung als Echtzeitsystem}
		\subsection{Verwendete Hardware}
			\subsubsection{EDMA-Verfahren}
			\subsubsection{Doppel-Puffer-Verfahren}
    
    \section{Ermittlung der Systemparameter}
        \subsection{Winkelauflösung}
        \subsection{EDMA-Länge}
        \subsection{Räumliches Aliasing}
    
    
\chapter{Test und Auswertung}




\chapter{Fazit / Ausblick}



